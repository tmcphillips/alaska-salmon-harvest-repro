\PassOptionsToPackage{unicode=true}{hyperref} % options for packages loaded elsewhere
\PassOptionsToPackage{hyphens}{url}
%
\documentclass[]{article}
\usepackage{lmodern}
\usepackage{amssymb,amsmath}
\usepackage{ifxetex,ifluatex}
\usepackage{fixltx2e} % provides \textsubscript
\ifnum 0\ifxetex 1\fi\ifluatex 1\fi=0 % if pdftex
  \usepackage[T1]{fontenc}
  \usepackage[utf8]{inputenc}
  \usepackage{textcomp} % provides euro and other symbols
\else % if luatex or xelatex
  \usepackage{unicode-math}
  \defaultfontfeatures{Ligatures=TeX,Scale=MatchLowercase}
\fi
% use upquote if available, for straight quotes in verbatim environments
\IfFileExists{upquote.sty}{\usepackage{upquote}}{}
% use microtype if available
\IfFileExists{microtype.sty}{%
\usepackage[]{microtype}
\UseMicrotypeSet[protrusion]{basicmath} % disable protrusion for tt fonts
}{}
\IfFileExists{parskip.sty}{%
\usepackage{parskip}
}{% else
\setlength{\parindent}{0pt}
\setlength{\parskip}{6pt plus 2pt minus 1pt}
}
\usepackage{hyperref}
\hypersetup{
            pdftitle={Harvest by Region, Species, and Sector},
            pdfauthor={Jeanette Clark},
            pdfborder={0 0 0},
            breaklinks=true}
\urlstyle{same}  % don't use monospace font for urls
\usepackage[margin=1in]{geometry}
\usepackage{longtable,booktabs}
% Fix footnotes in tables (requires footnote package)
\IfFileExists{footnote.sty}{\usepackage{footnote}\makesavenoteenv{longtable}}{}
\usepackage{graphicx,grffile}
\makeatletter
\def\maxwidth{\ifdim\Gin@nat@width>\linewidth\linewidth\else\Gin@nat@width\fi}
\def\maxheight{\ifdim\Gin@nat@height>\textheight\textheight\else\Gin@nat@height\fi}
\makeatother
% Scale images if necessary, so that they will not overflow the page
% margins by default, and it is still possible to overwrite the defaults
% using explicit options in \includegraphics[width, height, ...]{}
\setkeys{Gin}{width=\maxwidth,height=\maxheight,keepaspectratio}
\setlength{\emergencystretch}{3em}  % prevent overfull lines
\providecommand{\tightlist}{%
  \setlength{\itemsep}{0pt}\setlength{\parskip}{0pt}}
\setcounter{secnumdepth}{0}
% Redefines (sub)paragraphs to behave more like sections
\ifx\paragraph\undefined\else
\let\oldparagraph\paragraph
\renewcommand{\paragraph}[1]{\oldparagraph{#1}\mbox{}}
\fi
\ifx\subparagraph\undefined\else
\let\oldsubparagraph\subparagraph
\renewcommand{\subparagraph}[1]{\oldsubparagraph{#1}\mbox{}}
\fi

% set default figure placement to htbp
\makeatletter
\def\fps@figure{htbp}
\makeatother


\title{Harvest by Region, Species, and Sector}
\author{Jeanette Clark}
\date{5/23/2018}

\begin{document}
\maketitle

\hypertarget{datasets}{%
\section{Datasets}\label{datasets}}

The datasets used in this analysis are:

\hypertarget{subsistence-and-personal-use-harvest}{%
\subsection{\texorpdfstring{\href{https://knb.ecoinformatics.org/\#view/urn:uuid:7e4586c0-9812-4355-8f3b-1445b9a8ca53}{Subsistence
and Personal Use
Harvest}}{Subsistence and Personal Use Harvest}}\label{subsistence-and-personal-use-harvest}}

Fishing permits and postseason household surveys are used to assess
subsistence salmon harvests in the management areas in Alaska. Fishing
permits vary by subdistrict and are returned to the Alaska Department of
Fish and Game (ADFG) if not used. Household surveys are conducted by the
Division of Commercial Fisheries in certain subdistricts after the end
of the season. Postseason surveys are used to estimate subsistence
salmon by species and community; compile information on harvest by gear
types, participation rates, household size, use of salmon, and
participation in customary barter and trade; as well as assess the
quality of salmon fishing and what affected it.

Certain subsistence fisheries span multiple SASAP Regions. Cook Inlet
and Prince William Sound Educational Permits span two regions. The vast
majority of these permits are held by Cook Inlet residents, so this
fishery is assigned to Cook Inlet. The Northwest Alaska subsistence
fishery encompasses waters located both Norton Sound and Kotzebue. The
residency of permitholders in the Northwest Fishery is split between
Norton Sound and Kotzebue, with Norton Sound residents taking
\textasciitilde{}2/3 of the catch in this fishery. A tiny percentage of
the catch is taken by residents of other parts of the state. Since both
of these regions are remote and rural, it is reasonable to assume that
Kotzebue residents mostly fish in Kotzebue, and Norton Sound residents
mostly fish in Norton Sound. Thus, the region of origin of the fish was
assigned based on the residence of the permit holder for this fishery
only. The small percentage of the catch (\textless{} 5\% annually)
caught by residents of other parts of the state is assigned to Norton
Sound. With the exception of these two fisheries, the SASAP region is
assigned based on the location of the fishery.

\hypertarget{commercial-harvest}{%
\subsection{\texorpdfstring{\href{https://knb.ecoinformatics.org/\#view/urn:uuid:40473bde-9774-4581-aafb-5d2c3b4a70d1}{Commercial
Harvest}}{Commercial Harvest}}\label{commercial-harvest}}

Two sources are used for Commercial Harvest data. One dataset comes
directly from the Commercial Fisheries Entry Commision (CFEC), and shows
harvest at the district level, by species and gear type. The CFEC data
is derived from individual fish tickets. In this dataset, commercial
data with fewer than 3 people, permits, or processors in a fishery are
confidental and thus not shown.

A second data source, the official
\href{https://knb.ecoinformatics.org/\#view/urn:uuid:31b421f3-c48c-473a-bc20-601c738b3a3c}{ADFG
Annual Management Reports}, was incorporated where possible. These
reports have data for commercial catch by species at the management area
level. For various reasons, these data are far less susceptible to
confidentiality issues, and thus in some regions can fill in
considerable gaps left by the CFEC data. ADFG management report data was
compiled for all regions except Prince William Sound, Southeast, and
Arctic. The Arctic does not have any active commercial fisheries, so it
was not included. Prince William Sound and Southeast management reports
have difficult structures that made it difficult to extract data in the
same way as the CFEC data, so they were not included. These two regions
are also not very susceptible to confidentiality issues.

Where the two data sources were both available, and no CFEC data were
confidential, a two-sample t-test was conducted to test that the two
data sources were not significantly different. Across regions, the
sources are not significantly different. A slightly lower p-value exists
in the Yukon, where from 1995-1997 reported harvest values are higher
than CFEC data because the reported values include fish harvested for
roe, while CFEC data does not. Additionally, the reported Yukon harvest
values include Canadian harvest, which is not included in the CFEC data.

Since the ADFG reports all source the CFEC for their data, the harvest
numbers should not be significantly different between the two sources,
other than the noted differences in the Yukon, the differences between
the two datasets are not meaningful.

\begin{longtable}[]{@{}lr@{}}
\toprule
& p-value\tabularnewline
\midrule
\endhead
Alaska Peninsula and Aleutian Islands & 0.9817614\tabularnewline
Copper River & 0.9968430\tabularnewline
Kodiak & 0.8867265\tabularnewline
Kotzebue & 0.9669972\tabularnewline
Kuskokwim & 0.9345723\tabularnewline
Norton Sound & 0.9179088\tabularnewline
Chignik & 0.9845932\tabularnewline
Cook Inlet & 0.9620520\tabularnewline
Yukon & 0.3776158\tabularnewline
Bristol Bay & 0.9849001\tabularnewline
\bottomrule
\end{longtable}

Because the report data are more complete for this time period, and
there is no difference between the two datasets, data sourced from ADFG
reports is used instead of CFEC data where available. If CFEC data was
used, where more than 50\% of the commercial fisheries in a year had
confidential data due to lack of processors, and not people/permits, a
light gray bar is shown in the plots below. This criteria was chosen
because although there is no way of knowing how much of the total
harvest is represented in the confidential data, data that are
confidential due to lack of people/permits are likely to represent a
small fishery, whereas data that are confidential due to a lack of
processors could represent a very large fishery.

\hypertarget{sportfish-harvest}{%
\subsection{\texorpdfstring{\href{https://knb.ecoinformatics.org/\#view/urn:uuid:6a6a530f-3660-424f-adab-c771d1c89a5d}{Sportfish
Harvest}}{Sportfish Harvest}}\label{sportfish-harvest}}

Sport fish harvest is estimated by an annual mail survey conducted by
ADFG. These surveys are mailed to a stratified random sample of
households where at least one person had an Alaska sport fishing license
in that year. Harvest data are estimated for each site and species each
year by expanding the data according to the methods in
\href{https://knb.ecoinformatics.org/knb/d1/mn/v2/object/urn\%3Auuid\%3Ad64e5f8b-c91c-487a-8ce7-0cd271194f34}{Alaska
Statewide Sport Fish Harvest Suvey, Regional Operational Plan}.

Estimates that were based on fewer than 12 responses were not used in
this analysis, based on the recommendation of
\href{https://knb.ecoinformatics.org/knb/d1/mn/v2/object/urn\%3Auuid\%3Abb01b2c8-5e6c-4645-903d-39dbdd8d4d56}{Mills
and Howe (1992)}. This dataset only uses the point estimate for each
harvest value, although standard error, and upper and lower confidence
limits are available.

\hypertarget{figures}{%
\section{Figures}\label{figures}}

These plots show the harvest in thousands of fish by sector, species and
region. Region indicates where the fish were caught.

\includegraphics{All_Harvest_files/figure-latex/unnamed-chunk-8-1.png}
\includegraphics{All_Harvest_files/figure-latex/unnamed-chunk-8-2.png}
\includegraphics{All_Harvest_files/figure-latex/unnamed-chunk-8-3.png}
\includegraphics{All_Harvest_files/figure-latex/unnamed-chunk-8-4.png}
\includegraphics{All_Harvest_files/figure-latex/unnamed-chunk-8-5.png}
\includegraphics{All_Harvest_files/figure-latex/unnamed-chunk-8-6.png}
\includegraphics{All_Harvest_files/figure-latex/unnamed-chunk-8-7.png}
\includegraphics{All_Harvest_files/figure-latex/unnamed-chunk-8-8.png}
\includegraphics{All_Harvest_files/figure-latex/unnamed-chunk-8-9.png}
\includegraphics{All_Harvest_files/figure-latex/unnamed-chunk-8-10.png}
\includegraphics{All_Harvest_files/figure-latex/unnamed-chunk-8-11.png}
\includegraphics{All_Harvest_files/figure-latex/unnamed-chunk-8-12.png}
\includegraphics{All_Harvest_files/figure-latex/unnamed-chunk-8-13.png}

\end{document}
